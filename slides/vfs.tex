\documentclass{beamer}
\usetheme{Madrid}

% escreve textos gerados em portugues
\usepackage[brazilian]{babel}
% aceita unicode
\usepackage[utf8]{inputenc}

\author[Andre P. Esteve e Zhenlei Ji]{
Andre Petris Esteve - \texttt{andreesteve@gmail.com}\\
Zhenlei Ji - \texttt{zhenlei.ji@gmail.com}}
\institute[IC\textbackslash UNICAMP]{
MC806 - Tópicos em sistemas operacionais\\}

\title[Linux VFS]{Linux Virtual File System}
\subtitle[]{Introdução ao sistema virtual de arquivos}

\date[20/10/2011]{20 de outubro de 2011}

\begin{document}

%--- create section frame for every new section --%
\AtBeginSection[]
{
   \begin{frame}
       \frametitle{Agenda}
       \tableofcontents[currentsection]
   \end{frame}
}

\begin{frame}[plain]
  \titlepage
\end{frame}

\section{Exercício 1}

\begin{frame}{Exercício 1}

  \begin{block}{Exercício 1}

    \begin{itemize}

    \item{Escreva uma classe em java, chamada Lista, que permite inserção e remoção de valores inteiros.
Conforme os valores são inseridos ou removidos, o valor total da lista deve ser computado, de forma, que em qualquer
momento deve ser possível obter a soma dos valores na lista em O(1), ou seja, sem percorrer a lista.}

    \item{Para testar sua classe: na Main, instancie um novo objeto a partir da classe Lista, adicione e remova elementos. Também tente alterar o limite e verifique se o comportamento é o esperado.}

    \end{itemize}

  \end{block}

\end{frame}


%--- obrigado-------------------------------------%

\begin{frame}[plain]

  \begin{center}
    \Huge Dúvidas, comentários?
  \end{center}

  \vspace{0.2in}

  \begin{center}
	Andre Petris Esteve - \texttt{andreesteve@gmail.com}\\
	Zhenlei Ji - \texttt{zhenlei.ji@gmail.com}
  \end{center}
\end{frame}

\end{document}
