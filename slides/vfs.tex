\documentclass{beamer}
\usetheme{Madrid}

% escreve textos gerados em portugues
\usepackage[brazilian]{babel}
% aceita unicode
\usepackage[utf8]{inputenc}

\author[Andre Esteve and Zhenlei Ji]{
Andre Petris Esteve - \texttt{andreesteve@gmail.com}\\
Zhenlei Ji - \texttt{zhenlei.ji@gmail.com}}
\institute[IC\textbackslash UNICAMP]{
MC806 - Operational System Topics\\}

\title[Linux VFS]{Linux Virtual File System}
\subtitle[]{The linux VFS and FUSE - File System in User Space}

\date[10/20/2011]{October 20th, 2011}

\begin{document}

%--- create section frame for every new section --%
\AtBeginSection[]
{
   \begin{frame}
       \frametitle{Agenda}
       \tableofcontents[currentsection]
   \end{frame}
}

\begin{frame}[plain]
  \titlepage
\end{frame}

%--- content -------------------------------------%
\begin{frame}{Agenda}
  \tableofcontents
\end{frame}

\section{Linux's Virtual File System Overview}

\begin{frame}{What's Linux's Virtual File System}

  \begin{block}{Definition}

	The Virtual File System (also known as the Virtual Filesystem Switch)
	is the software layer in the kernel that provides the filesystem
	interface to userspace programs. It also provides an abstraction
	within the kernel which allows different filesystem implementations to
	coexist. \footnotemark

  \end{block}

	\footnotetext[1]{Overview of the Linux Virtual File System, Linux "documentation", Richard Gooch}

\end{frame}


%--- obrigado-------------------------------------%

\begin{frame}[plain]

  \begin{center}
    \Huge Questions?
  \end{center}

  \vspace{0.2in}

  \begin{center}
	Andre Petris Esteve - \texttt{andreesteve@gmail.com}\\
	Zhenlei Ji - \texttt{zhenlei.ji@gmail.com}
  \end{center}
\end{frame}

\end{document}
